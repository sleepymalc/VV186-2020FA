\documentclass[12pt, t]{beamer}
\usepackage{graphicx}
\usepackage{amsmath}
\usepackage{setspace}
\usepackage{float} 
\usepackage{multido}
\usepackage{multirow}
\usepackage{array}
\usepackage{enumerate}
\usepackage{booktabs}
\usepackage{indentfirst} 
\usepackage[style=mla]{biblatex}
\usepackage{subcaption}
\usepackage{hyperref}
\usepackage{textpos}
\usepackage{mathtools, nccmath}

\makeatletter
\let\@@magyar@captionfix\relax
\makeatother

\definecolor{Turquoise3}{RGB}{0, 134, 139}
\renewcommand{\emph}[1]{{\color{Turquoise3}\textsl{#1}}}
\newcommand{\C}{\mathbb{C}} \newcommand{\F}{\mathbb{F}} \newcommand{\R}{\mathbb{R}} \newcommand{\Q}{\mathbb{Q}}
\newcommand{\N}{\mathbb{N}}
\newcommand{\myseries}[2]{$#1_1,#1_2,\dots,#1_#2$}
\newcommand{\nullspace}{~\\[15pt]}
\newcommand{\remark}{\textbf{Remark: }}
\newcommand{\scp}[2]{\langle\,#1\,,\,#2\,\rangle} \newcommand{\scpp}{\langle\,\cdot\,,\,\cdot\,\rangle}


\usetheme{Madrid}
\setbeamertemplate{navigation symbols}{}

\addtobeamertemplate{frametitle}{}{
\begin{textblock*}{100mm}(0.85\textwidth,-1cm)
\includegraphics[height=1cm]{Figures/logo/logo.png}
\end{textblock*}}

\definecolor{themecolor}{RGB}{25,25,112} 

\usecolortheme[named=themecolor]{structure}

\setbeamertemplate{items}[default]

\hypersetup{
    colorlinks=true,
    linkcolor=themecolor,
    filecolor=themecolor,      
    urlcolor=themecolor,
    citecolor=themecolor,
}

\title{VV186 RC Part III}
\subtitle{\textbf{Convergence}\\``Find the problem, understand it, then explore more."}
\institute[UM-SJTU JI]{University of Michigan-Shanghai Jiao Tong University Joint Institute}
\author{Pingbang Hu}

\begin{document}

\begin{frame}
    \titlepage
    \begin{center}
        \includegraphics[height=2cm]{Figures/logo/logo2.png}
    \end{center}
\end{frame}

\begin{frame}
    \frametitle{Overview}
    \begin{enumerate}
        \item Review
        \item Metric Space
        \item Cauchy Sequence
        \item Generalization of Convergence
        \item Completeness
        \item Construct Real Numbers
        \item Real Functions
        \item Exercise
        \item More about Cauchy Sequence
    \end{enumerate}
\end{frame}

\section{Review}
\begin{frame}
    \frametitle{Review}
    Let first look at the Exercise 7 from last time. \\
    \vspace{1em}
    Let $(x_n)$ be a bounded real sequence. Then define
    \begin{equation*}
        a_n:=\sup_{m\geq n}(x_m),\quad b_n:=\inf_{m\geq n}(x_m)
    \end{equation*}

    \begin{enumerate}
        \item Prove that $(a_n)$ is decreasing, while $(b_n)$ is increasing.
        \item Since both $(a_n),(b_n)$ are monotonic and bounded, they are convergent.
              We denote $\underline{\lim} x_n=\lim b_n$; $\overline{\lim}x_n=\lim a_n$. Show that:
              \begin{equation*}
                  \underline{\lim}y_n+\underline{\lim}z_n\leq \underline{\lim} (y_n+z_n)\leq\overline{\lim}y_n+\underline{\lim}z_n\leq\overline{\lim}y_n+\overline{\lim}z_n
              \end{equation*}
    \end{enumerate}

\end{frame}

\section{Stop}
\begin{frame}
    \frametitle{This is a title!}
    \vspace{6em}
    \begin{center}
        \Large{Any Questions?}
    \end{center}

\end{frame}

\section{Metric Space}
\begin{frame}
    \frametitle{Metric Space}
    \begin{itemize}
        \item What is the definition of a metric?
        \item Why we want to introduce the idea of Metric Space?
        \item What new results can we explore from this new idea?
    \end{itemize}
\end{frame}

\section{Metric Space}
\begin{frame}
    \frametitle{Metric Space}
    We want to generalize the idea of \emph{convergence}, we want to define the most essential thing of convergence by ourselves,
    namely the \textbf{Length Function}.\\

    \vspace{1em}

    What properties a usual length function should have?
    \begin{enumerate}
        \item Always positive.
        \item Symmetric.
        \item Followed \emph{Triangle Inequality}.
    \end{enumerate}

    \vspace{1em}

    Transform these into mathematical language\dots
\end{frame}

\section{Metric Space}
\begin{frame}
    \frametitle{Metric Space}
    A two variables functions

    \begin{equation*}
        \rho(\cdot,\cdot):M\times M \rightarrow \mathbb{R}
    \end{equation*}

    is called a metric if it satisfies:

    \begin{enumerate}
        \item $\forall x,y\in M,\ \rho (x,y) \geq 0$ and $\rho (x,y)=0$ if and only if $x=y$.
        \item  $\forall x,y\in M,\ \rho (x,y)=\rho (y,x)$.
        \item  $\forall x,y,z\in M,\ \rho (x,z)\leq \rho (x,y)+\rho (y,z)$.
    \end{enumerate}


\end{frame}

\section{Metric Space}
\begin{frame}
    \frametitle{Examples}
    \begin{itemize}
        \item $M=\mathbb{R}^n$, the usual metric is given by
              \begin{equation*}
                  \rho ( (x_1,x_2,\dots,x_n), (y_1,y_2,\dots,y_n)) = \sqrt{\sum^{n}_{i=k}(x_i-y_i)^2 }
              \end{equation*}
              and this is so-called \emph{Euclidean distance}.
        \item $M=\mathbb{N}, \rho(x,y)= \#\{ a:a\in [min\{x,y\},max\{x,y\}]\ \} $
        \item $M=\mathbb{R},\rho (x,y)=1 $ if $x\neq y$; $\rho (x,y)=0$ if $x=y$
    \end{itemize}
    \vspace{1em}
    A simple question arise, is any metric \textbf{well-defined}?
\end{frame}

\section{Generalization of Convergence}
\begin{frame}
    \frametitle{Generalization of Convergence}
    Then, by replacing the usual matric $\rho(x,y)=|x-y|$ and choosing our universal set $M$, we get the natural definition for
    generalize convergence in metric space $(M,\rho)$ for a sequence $(a_n):\mathbb{N}\rightarrow M$, which is given by:
    \begin{equation*}
        \lim_{n\rightarrow \infty}a_n=a\quad :\Leftrightarrow \quad \underset{\epsilon>0}{\forall}\ \underset{N\in \mathbb{N}}{\exists}\ \underset{n>\mathbb{N}}{\forall} a_n\in B_\epsilon(a)
    \end{equation*}
    where
    \begin{equation*}
        B_\epsilon(a)=\{ x\in M:\rho(x,a)<\epsilon\},\quad \epsilon>0,\quad a\in M.
    \end{equation*}

\end{frame}

\section{Cauchy Sequence}
\begin{frame}
    \frametitle{Cauchy Sequence}
    \begin{itemize}
        \item What is the definition of Cauchy Sequence?
        \item How to understand Cauchy Sequence?
        \item Why we want to introduce the idea of Cauchy Sequence?
        \item What new results can we explore from this new idea?
    \end{itemize}

\end{frame}

\section{Cauchy Sequence}
\begin{frame}
    \frametitle{Cauchy Sequence}
    The fundamental reason why we want to introduce Cauchy Sequence is because we want to \emph{further generalize} the idea of
    convergence.\\
    \vspace{0.5em}
    Now, we are not only free to choose the metric we like, additionally, we remove the constraint which requires a sequence
    to converge to a \emph{specific point}.

\end{frame}

\section{Cauchy Sequence}
\begin{frame}
    \frametitle{Cauchy Sequence}
    We list some important results and theorems for Cauchy Sequence.
    \begin{itemize}
        \item Every convergent sequence is a Cauchy sequence.
        \item Every Cauchy sequence in a metric space $(M,\rho)$ is bounded.
        \item Every Cauchy sequence in $\mathbb{R}$ with the usual metric is convergent.
    \end{itemize}
\end{frame}

\section{Completeness}
\begin{frame}
    \frametitle{Digest\dots}
    We now take a deep breath, and look back what we have introduced. Where does all these new definitions lead us to?\\

\end{frame}

\section{Completeness}
\begin{frame}
    \frametitle{Completeness}
    The problem is, a Cauchy Sequence can simply not "\emph{converge}" to anywhere anymore if we generalize the idea of convergence.\\
    \vspace{0.5em}
    Then, after generalizing the idea of convergence, if \textbf{every} Cauchy sequence still converge, we say this metric space is
    \emph{complete}.\\
    \vspace{0.5em}
    We now take a look at some examples for a Cauchy sequence failed to converge.

\end{frame}

\section{Completeness}
\begin{frame}
    \frametitle{Completeness}
    For a metric space $(\mathbb{Q},\rho)$, where $\rho(x,y)$ is analogous to our usual metric. Then, a sequence
    \begin{equation*}
        (a_n)_{n\in\mathbb{N}}:=1,1.4,1.41,1.414,\dots
    \end{equation*}
    is a Cauchy sequence but failed to converge in $\mathbb{Q}$ if we choose the following terms appropriately and let it \emph{converge
        to $\sqrt{2}$ in $\mathbb{R}$} eventually.\\
    \vspace{0,5em}

    From here, it seems we somehow find a good way to \emph{construct} Real Numbers.
\end{frame}

\section{Construct Real Numbers}
\begin{frame}
    \frametitle{Construct Real Numbers}
    The main idea to \textbf{complete} $\mathbb{Q}$ is to consider the following\\
    \begin{center}
        Every Cauchy sequence can \emph{represent} a number in $\mathbb{R}$.
    \end{center}

    For example, from the last slide, we can view the sequence
    \begin{equation*}
        (a_n)_{n\in\mathbb{N}}=1,1.4,1.41,1.414,\dots
    \end{equation*}
    as $\sqrt{2}$ by defining a \emph{(equivalence) class} for each number in $\mathbb{R}$, and in this case, we define the sequence $(a_n)$ as $\sqrt{2}$.\\
    \vspace{0.5em}
    Then, from now on, the number defined through the Cauchy sequence which is given by
    \begin{equation*}
        0.9,0.99,0.999,0.9999,0.99999
    \end{equation*}
    is equal to $1$, namely $0.999999_{\dots}=1$.
\end{frame}

\section{Real Functions}
\begin{frame}
    \frametitle{Real Functions}
    We already know how to describe a function, so we list some \emph{elementary functions} and some categories of functions
    which you will encounter a lot in the future.
    \begin{itemize}
        \item Polynomial Function : $f(x)=\sum^n_{k=0}c_kx^k,\ n\in \mathbb{N}.\ \underset{1\leq i\leq n}{\forall}c_i\in \mathbb{R}$
        \item Power Function : $f(x)=x^n,\ n\in \mathbb{N}$
        \item Rational Function : $f(x)=\frac{p(x)}{q(x)},\ p,q\in \mathbb{P}(\mathbb{R})$
        \item Periodic Function : $\underset{x\in \mathbb{R}}{\forall}\ f(x+T)=f(x)$, where $T$ is \emph{period}.
        \item Piecewise Function
    \end{itemize}
    (What is the domain of each kind of function?)
\end{frame}

\section{Real Functions}
\begin{frame}
    \frametitle{Real Functions}
    Then, we can do some basic manipulation to a function, including:
    \begin{enumerate}
        \item Translation in different directions
        \item Multiplying a function's argument
        \item Multiplying a function's values
    \end{enumerate}
\end{frame}

\section{Exercise}
\begin{frame}
    \frametitle{Exercise}
    1. Suppose the function which is given by
    \begin{equation*}
        f:[2,+\infty)\rightarrow \mathbb{R}
    \end{equation*}
    and satisfies
    \begin{equation*}
        x^2f(x)-f(1-x)=x
    \end{equation*}
    Please calculate the value of
    \begin{equation*}
        \lim_{x\rightarrow \infty}\frac{f(x)}{3}
    \end{equation*}
    (Can we further expand the domain of f?)
\end{frame}

\section{Exercise}
\begin{frame}
    \frametitle{Exercise}
    2. Please construct a metric that makes $\mathbb{R}$ incomplete with regard to this metric. You can use the method given in
    Horst's slides.
\end{frame}

\section{More about Cauchy Sequence}
\begin{frame}
    \frametitle{More About Cauchy Sequence}
    Let us take a look at some tricky examples of Cauchy sequence.\\
    \vspace{5em}
    \begin{center}
        \Large \emph{In class demonstration!}
    \end{center}
\end{frame}

\begin{frame}
    \frametitle{End}
    \vspace{2cm}
    \Huge \center  Have Fun and Learn Well!
\end{frame}

\end{document}
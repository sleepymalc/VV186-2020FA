\documentclass[12pt, t]{beamer}
\usepackage{graphicx}
\usepackage{amsmath}
\usepackage{setspace}
\usepackage{float} 
\usepackage{multido}
\usepackage{multirow}
\usepackage{array}
\usepackage{enumerate}
\usepackage{booktabs}
\usepackage{indentfirst} 
\usepackage[style=mla]{biblatex}
\usepackage{subcaption}
\usepackage{hyperref}
\usepackage{textpos}
\usepackage{mathtools, nccmath}
\usepackage{amssymb}

\makeatletter
\let\@@magyar@captionfix\relax
\makeatother

\definecolor{Turquoise3}{RGB}{0, 134, 139}
\renewcommand{\emph}[1]{{\color{Turquoise3}\textsl{#1}}}
\newcommand{\C}{\mathbb{C}} \newcommand{\F}{\mathbb{F}} \newcommand{\R}{\mathbb{R}} \newcommand{\Q}{\mathbb{Q}}\newcommand{\N}{\mathbb{N}}
\newcommand{\myqed}{\hfill$\square$}
\newcommand{\myseries}[2]{$#1_1,#1_2,\dots,#1_#2$}
\newcommand{\nullspace}{~\\[15pt]}
\newcommand{\remark}{\textbf{Remark: }}
\newcommand{\scp}[2]{\langle\,#1\,,\,#2\,\rangle} \newcommand{\scpp}{\langle\,\cdot\,,\,\cdot\,\rangle}


\usetheme{Madrid}
\setbeamertemplate{navigation symbols}{}

\addtobeamertemplate{frametitle}{}{
\begin{textblock*}{100mm}(0.85\textwidth,-1cm)
\includegraphics[height=1cm]{Figures/logo/logo.png}
\end{textblock*}}

\definecolor{themecolor}{RGB}{25,25,112} 

\usecolortheme[named=themecolor]{structure}

\setbeamertemplate{items}[default]

\hypersetup{
    colorlinks=true,
    linkcolor=themecolor,
    filecolor=themecolor,      
    urlcolor=themecolor,
    citecolor=themecolor,
}

\title{VV186 RC Part II Solution}
\subtitle{\textbf{From Numbers to Sequences -- Exercises}\\``Practice makes Perfect\dots"}
\institute[UM-SJTU JI]{University of Michigan-Shanghai Jiao Tong University Joint Institute}
\author{Pingbang Hu}

\begin{document}

\begin{frame}
    \titlepage
    \begin{center}
        \includegraphics[height=2cm]{Figures/logo/logo2.png}
    \end{center}
\end{frame}

\begin{frame}
    \frametitle{Exercises-I}
    1. When learning the axioms of rational number, one student found that the operation of subsets of a non-empty
    set $X$ is somewhat similar to that of rational number:\\
    \vspace{1em}
    If we regard $\cup$ as $+$, $\cap$ as $\cdot$, then the equation
    \begin{equation*}
        A\cap(B\cup C)=(A\cap B)\cup(A\cap C)
    \end{equation*}
    is just the distributivity law. Help him  check whether P1 -- P9 also hold for such operations.
\end{frame}

\begin{frame}
    \frametitle{Solution-I}
    We check by enumerating P1 -- P9.
    \begin{enumerate}
        \item $A\cup(B\cup C) = (A\cup B)\cup C$
        \item $A\cup \emptyset =\emptyset \cup A=A$
        \item $A\cup B=B\cup A$
        \item \emph{$A^c\cup A=A\cup A^c=X$}
        \item $A\cap(B\cap C) = (A\cap B)\cap C$
        \item $A\cap X = X\cap A=A$
        \item $A\cap B=B\cap A $
        \item \emph{$A\cap \emptyset=\emptyset \cap A=\emptyset$}
        \item $A\cap (B\cup C)=(A\cap B)\cup (A\cap C)$
    \end{enumerate}
    As we can see, P4 and P8 does not hold, since we can't find the inverse element for $\cup$ and $\cap$, so the answer of this
    exercise is no.
    \myqed
\end{frame}

\begin{frame}
    \frametitle{Exercises-I}
    2.
    \begin{enumerate}
        \item Prove that for $a_i\in \mathbb{Q}, i\in \mathbb{N^*}$, and $n$ is a natural number,
              \begin{equation*}
                  |\sum^n_{i=1}a_i|\leq\sum^n_{i=1}|a_i|
              \end{equation*}
        \item Prove that $|a-c|\leq |a-b| +|c-b|$
    \end{enumerate}
\end{frame}

\begin{frame}
    \frametitle{Solution-I}
    2 -- 1\\
    $Proof$ :
    \begin{equation*}
        \begin{split}
            |\sum^n_{i=1}a_i| &=|a_1+a_2+\dots+a_n|
            =|(a_1+\dots+a_{n-1})+a_n|\\
            &\leq |a_1+\dots+a_{n-1}|+|a_n|
            \leq \dots\\
            &.\\
            &.\\
            &.\\
            &\leq |a_1|+|a_2|+\dots+|a_n|=\sum^n_{i=1}|a_i|
        \end{split}
    \end{equation*}
    \myqed
\end{frame}

\begin{frame}
    \frametitle{Solution-I}
    2 -- 2\\
    $Proof$ :
    \begin{equation*}
        \begin{split}
            |a-c|&=|a+b-b-c|\\
            &=|(a-b)+(b-c)|\\
            &\leq|a-b|+|b-c|\\
            &=|a-b|+|c-b|
        \end{split}
    \end{equation*}
    \myqed
\end{frame}


\begin{frame}
    \frametitle{Exercises-I}
    3. We define $\mathbb{R}$ as a set containing rational numbers and the limits of rational numbers.
    Let $E\subseteq \mathbb{R}$ a non-empty subset of that is bounded above in $\mathbb{R}$. Now we don't
    assume P13 is an axiom. We want to prove P13 with P1 -- P12 and other facts that we know:
    \begin{itemize}
        \item Let's call the set of form $(a,+\infty)$ or $[a,+\infty)$ \emph{up interval}. Prove that the
              set of $E$'s upper bound, denoted by $U$, is an \emph{up interval}.
        \item (\textbf{Theorem 2.2}) $E$ has a least upper bound.
    \end{itemize}
\end{frame}

\begin{frame}
    \frametitle{Solutions-I}
    3 -- 1\\
    $Proof$ : \\
    \hspace{1em} We first prove that $U$ is an interval by contradiction. Suppose U is not an interval, then we can see
    \begin{equation*}
        \exists\ m\notin U\
        \begin{cases}
            U\cap(-\infty, m)\neq \emptyset \\
            U\cap(m,\infty)\neq \emptyset
        \end{cases}
    \end{equation*}
    \hspace{1em} From here, we have two useful conditions as follows.
    \begin{enumerate}
        \item Since $m\notin U$, $\underset{x\in E}{\exists}\ x>m $.
        \item Since $(-\infty,m)\cap U\neq \emptyset$, $\exists\ y\in (-\infty,m)\cap U\Rightarrow
                  \begin{cases}
                      y<m \\
                      y\in U
                  \end{cases}$
    \end{enumerate}
    \hspace{1em} Finally, we get $y<m<x$, but this is a contradiction since $y$ is an upper bound but $y<x\in E$.
\end{frame}

\begin{frame}
    \frametitle{Solutions-I}
    3 -- 1\\
    $Proof(continue)$ : \\
    \hspace{1em} Then, we know U is an interval now. Because $U$ is clearly bounded below by some element in $E$, let say $a$, then
    $U$ should have this form:
    \begin{equation*}
        U=(a,b)\ or\ U=[a,b)
    \end{equation*}
    for some $b$. And because if $b\in U$, then clearly $b+c\in U$ for $c>0$, we know that
    \begin{equation*}
        U=(a,\infty)\ or\ U=[a,\infty)
    \end{equation*}
    so U is an up interval.
    \myqed
\end{frame}

\begin{frame}
    \frametitle{Solutions-I}
    3 -- 2\\
    $Proof$ : \\
    We consider two cases.\\
    First, if $U=[a,\infty)$, then clearly, $U$ has a minimum $a$, and which is the least upper bound of $E$.\\
    Second, if $U=(a,\infty)$, then we know
    \begin{equation*}
        \begin{cases}
            \underset{x\in E}{\exists}\ x>a\Rightarrow\ \underset{\epsilon>0}{\exists}\ x=a+2\epsilon \\
            \underset{\epsilon>0}{\forall}\ a+\epsilon\in U\ \Rightarrow\ a+\epsilon>x
        \end{cases}
    \end{equation*}
    We simply fixed the $\epsilon$ we found in equation 1, and then plug in to the equation 2, we get
    \begin{equation*}
        a+\epsilon>x=a+2\epsilon\ \Rightarrow\ \epsilon<0
    \end{equation*}
    Which contradicts to our chosen condition of $\epsilon$, so we know the second case $U=(a,\infty)$ can not be the case.
    \myqed
\end{frame}


\begin{frame}
    \frametitle{Exercises-I$^*$}
    4. Let $A$ be bounded set in $\mathbb{R}$ (which means that the total set is $\mathbb{R}$), for any $\epsilon>0$,
    there is an element $x$ in $A$ such that $|x-supA|<\epsilon$.
\end{frame}

\begin{frame}
    \frametitle{Solution-I}
    4\\
    $Proof$ : \\
    \hspace{1em} We proceed our proof by contradiction. Suppose this is not true, namely there is an $\epsilon>0$ such
    that for all $x\in A$ we have
    \begin{equation*}
        |x-\sup A|\geq \epsilon
    \end{equation*}
    But then we know, $\sup A-\frac{1}{2}\epsilon$ is an upper bound of $A$, which leads to a contradiction.
    \myqed
\end{frame}

\begin{frame}
    \frametitle{Exercise-II}
    5. A sequence is defined as
    \begin{equation*}
        (S_n)_{n\in\mathbb{N}},\ S_1=\sqrt{2},\ S_2=\sqrt{2\sqrt{2}},\ S_3=\sqrt{2\sqrt{2\sqrt{2}}}
    \end{equation*}
    Please calculate the limit of $(S_n)$ as $n\rightarrow \infty$, if it exists.
\end{frame}

\begin{frame}
    \frametitle{Solution-I}
    5.\\
    \hspace{1em} Since we don't know whether the limit of $(S_n)$ exist or not, we must first prove the existence, then do the following
    calculation.\\
    \hspace{1em} We proceed as follows. First, we observe that $(S_n)$ is bounded since
    \begin{equation*}
        0\leq S_n=\underbrace{\sqrt{2\sqrt{2\sqrt{2\dots\sqrt{2}}}}}_{n\ square\ roots} \leq \underbrace{\sqrt{2\sqrt{2\sqrt{2\dots 2}}}}_{n\ square\ roots}=2
    \end{equation*}
    We get the right-hand side by replacing the last $\sqrt{2}$ by $2$
    from the left-hand side.\\
    \hspace{1em} We conclude that the sequence $(S_n)$ is bounded. Our goal is to prove $(S_n)$ is convergent, so we now only need to prove
    $(S_n)$ is monotonic, then we can deduce the convergence from our already known lemma.
\end{frame}

\begin{frame}{Solution-I}
    $(continue)$\\
    \hspace{1em} Since $S_{n+1}=\sqrt{2S_n}$ for $n\geq 1$. When $n\geq 1$,
    \begin{equation*}
        \frac{S_{n+1}}{S_n}=\frac{\sqrt{2Sn}}{Sn}=\frac{\sqrt{2}}{\sqrt{S_n}}>\frac{\sqrt{2}}{\sqrt{2}}=1
    \end{equation*}
    \hspace{1em} It is easy to see that $(S_n)$ is strictly increasing, which means it is monotonic,
    hence convergent.\\
    \hspace{1em} To calculate the limit, we use the equation
    \begin{equation*}
        \lim_{n\rightarrow\infty} S_{n+1}=\lim_{n\rightarrow\infty} S_n =: L
    \end{equation*}
    \hspace{1em} Which leads to $\sqrt{2L}=L$. By solving this equation and the fact that $S_n\geq S_1=1$, so $L=2$, which mean the limit is 2.
    \myqed
\end{frame}


\begin{frame}
    \frametitle{Exercise-II}
    6. Let $(a_n),(b_n)$ be two real sequences. Furthermore, assume that $a_n<b_n$
    for all $n$, $[a_{n+1}, b_{n+1}]\subseteq [a_n,b_n]$, $\lim (a_n-b_n)=0$. Prove that there
    is an unique $m\in [a_n,b_n]$ for all $n$, such that
    \begin{equation*}
        \lim a_n=\lim b_n=m
    \end{equation*}
\end{frame}

\begin{frame}
    \frametitle{Solution-II}
    6.\\
    $Proof$ : \\
    \hspace{1em} We first show that $(a_n),(b_n)$ are convergent. Notice that $[a_{n+1}, b_{n+1}]\subseteq [a_n,b_n]$
    means that $(a_n)$ is increasing, while $(b_n)$ is decreasing. Furthermore, both $(a_n)$ and $(b_n)$ are bounded by
    $[a_0,b_0]$. Therefore, $(a_n)$ and $(b_n)$ are convergent.\\
    \hspace{1em} Next we show that $\lim a_n=\lim b_n$. Suppose $\lim b_n=L$, where $L$ is a unique number since a sequence
    has precisely one limit. Then we know
    \begin{equation*}
        \lim a_n=\lim [(a_n-b_n)+b_n]=\lim (a_n-b_n) + \lim b_n= 0+L=L
    \end{equation*}
    Moreover, since
    $\begin{cases}
            L \geq a_n \\
            L \leq b_n
        \end{cases}$, we know that $L\in [a_n,b_n]$.
    \myqed
\end{frame}

\begin{frame}
    \frametitle{Exercise-II}
    7. Let $(x_n)$ be a bounded real sequence. Then define
    \begin{equation*}
        a_n:=\sup_{m\geq n}(x_m),\quad b_n:=\inf_{m\geq n}(x_m)
    \end{equation*}

    \begin{enumerate}
        \item Prove that $(a_n)$ is decreasing, while $(b_n)$ is increasing.
        \item Since both $(a_n),(b_n)$ are monotonic and bounded, they are convergent.
              We denote $\underline{\lim} x_n=\lim b_n$; $\overline{\lim}x_n=\lim a_n$. Show that:
              \begin{equation*}
                  \underline{\lim}y_n+\underline{\lim}z_n\leq \underline{\lim} (y_n+z_n)\leq\overline{\lim}y_n+\underline{\lim}z_n\leq\overline{\lim}y_n+\overline{\lim}z_n
              \end{equation*}
    \end{enumerate}
\end{frame}

\begin{frame}
    \frametitle{Solution-II}
    7 -- 1\\
    $Proof$ : \\
    \hspace{1em} This follows from the fact that for two bounded sets $A,B$ in $\mathbb{R}$ with $A\supseteq B$, the supremum of $A$ is no less than
    the supremum of $B$.  Similarly, for two bounded sets $A,B$ in $\mathbb{R}$ with $A\supseteq B$, the infimum of $A$ is no greater than the infimum of $B$.


\end{frame}

\begin{frame}
    \frametitle{Solution-II}
    7 -- 2\\
    $Proof$ : \\
    \hspace{1em} We precede from the left-hand side. Start from the first inequality:
    \begin{equation*}
        \underline{\lim}y_n+\underline{\lim}z_n\leq \underline{\lim} (y_n+z_n)
    \end{equation*}
    \hspace{1em} Fix $\epsilon>0$, there is an $M\in \mathbb{N}$ such that $\underset{n>M}{\forall}\ y_n\geq \underline{\lim}y_n-\frac{\epsilon}{4}$, and
    $z_n \geq \underline{\lim}z_n-\frac{\epsilon}{4}$. So we have $\underline{\lim}y_n+\underline{\lim}z_n-\frac{\epsilon}{2}\leq(y_n+z_n)$ for all such $n$.
    Next, we claim that there is always a subsequence of $(y_n+z_n)$ such that it converges to $\underline{\lim}(y_n+z_n)$.\\
    \hspace{1em} Fix arbitrary $t>0$, $\underset{M\in\mathbb{N}}{\forall}\ \underset{n_k\geq M}{\exists}$ such that
    \begin{equation*}
        \begin{cases}
            |(y_{n_k}+z_{n_k})-\inf_{m\geq M}(y_m+z_m)|<\frac{1}{2}t   \\
            |(y_{n_k}+z_{n_k})-\underline{\lim}(y_n+z_n)|<\frac{1}{2}t \\
        \end{cases}
    \end{equation*}


\end{frame}

\begin{frame}
    \frametitle{Solution-II}
    7 -- 2\\
    $Proof(continue)$ : \\
    \hspace{1em} We know $|(y_{n_k}+z_{n_k})-\underline{\lim}(y_n-z_n)|<t$. This means for large enough $n_k$ (at least $n_k>M$) we make sure that
    $(y_{n_k}+z_{n_k})\leq \underline{\lim}(y_n+z_n)+\frac{\epsilon}{2}$ and also
    \begin{equation*}
        \underline{\lim}y_n+\underline{\lim}z_n-\frac{\epsilon}{2}\leq (y_{n_k}+z_{n_k})\leq \underline{\lim}(y_n+z_n)+\frac{\epsilon}{2}
    \end{equation*}
    \begin{equation*}
        \Rightarrow \underline{\lim}y_n+\underline{\lim}z_n\leq \underline{\lim}(y_n+z_n)+\epsilon
    \end{equation*}
    \hspace{1em} Since this is true for all $\epsilon>0$, we conclude that
    \begin{equation*}
        \underline{\lim}y_n+\underline{\lim}z_n\leq \underline{\lim}(y_n+z_n)
    \end{equation*}

\end{frame}

\begin{frame}
    \frametitle{Solution-II}
    7 -- 2\\
    $Proof(continue)$ : \\
    \hspace{1em} Now we prove the second inequality.
    \begin{equation*}
        \underline{\lim} (y_n+z_n)\leq\overline{\lim}y_n+\underline{\lim}z_n
    \end{equation*}
    \hspace{1em} From the fact that
    \begin{equation*}
        -\underline{\lim}(-x_n)=\overline{\lim}x_n
    \end{equation*}
    \hspace{1em} This equality holds because $-\inf_{m\geq n}(-x_m)=\sup_{m\geq n}x_m$. Now,
    $\underline{\lim}(y_n+z_n)+\underline{\lim}(-y_n)\leq \underline{\lim}(y_n+z_n+(-y_n))=\underline{\lim}z_n$
    by what we have proved above. Thus,
    \begin{equation*}
        \underline{\lim}(y_n+z_n)\leq -\underline{\lim}(-y_n)+\underline{\lim}z_n=\overline{\lim}y_n+\underline{\lim}z_n
    \end{equation*}

\end{frame}


\begin{frame}
    \frametitle{Solution-II}
    7 -- 2\\
    $Proof(continue)$ : \\
    \hspace{1em} Finally, we now prove the last inequality.
    \begin{equation*}
        \overline{\lim}y_n+\underline{\lim}z_n\leq\overline{\lim}y_n+\overline{\lim}z_n
    \end{equation*}
    \hspace{1em} Since $a_n:=\sup_{m\geq n}x_m\geq \inf_{m\geq n}x_m=:b_n$, we have
    \begin{equation*}
        \underline{\lim}z_n\leq \overline{\lim}z_n
    \end{equation*}
    \hspace{1em} Therefore,
    \begin{equation*}
        \overline{\lim}y_n+\underline{\lim}z_n\leq\overline{\lim}y_n+\overline{\lim}z_n
    \end{equation*}
    \myqed
\end{frame}




\begin{frame}
    \frametitle{Exercise-II$^*$}
    8. Let $(a_n)$ be a sequence such that
    \begin{equation*}
        a_n=\frac{1}{\sqrt{n^2+1}}+\dots+\frac{1}{\sqrt{n^2+n}}
    \end{equation*}
    Calculate the limit of $(a_n)$.
\end{frame}

\begin{frame}
    \frametitle{Solution-II$^*$}
    8.\\
    \hspace{1em} We first consider two sequence $(b_n)_{n\geq 1},(c_n)_{n\geq 1}$ given by
    \begin{equation*}
        \begin{cases}
            b_n:=\frac{1}{\sqrt{n^2}}+\dots+\frac{1}{\sqrt{n^2}} \\
            c_n:=\frac{1}{\sqrt{N^2+n}}+\dots+\frac{1}{\sqrt{N^2+n}}
        \end{cases}
    \end{equation*}
    \hspace{1em} Clearly,
    $\begin{cases}
            (b_n)\leq (a_n) \\
            (c_n)\geq (a_n)
        \end{cases}$
    for all $n\geq 1$. \\
    \hspace{1em} Then one can easily find out the limit for both $(b_n)$ and $(c_n)$, which can be calculated as
    \begin{equation*}
        \lim_{n\rightarrow\infty}b_n=\lim_{n\rightarrow \infty}\sum^n_{k=1}\frac{1}{\sqrt{n^2}}=\lim_{n\rightarrow \infty}\sum^n_{k=1}\frac{1}{n}=\lim_{n\rightarrow\infty}\frac{n}{n}=\lim_{n\rightarrow\infty}1=1
    \end{equation*}
\end{frame}

\begin{frame}
    \frametitle{Solution-II$^*$}
    8.$(continue)$\\
    \begin{equation*}
        \lim_{n\rightarrow\infty}c_n=\lim_{n\rightarrow \infty}\sum^n_{k=1}\frac{1}{\sqrt{n^2+n}}=\lim_{n\rightarrow \infty}\frac{n}{\sqrt{n^2+n}}=\lim_{n\rightarrow\infty}\frac{1}{\sqrt{1+\frac{1}{n}}}=1
    \end{equation*}
    \hspace{1em} By Squeeze Theorem, we know
    \begin{equation*}
        \lim_{n\rightarrow\infty}b_n\leq \lim_{n\rightarrow\infty}a_n\leq \lim_{n\rightarrow\infty}c_n
    \end{equation*}
    , namely
    \begin{equation*}
        1\leq \lim_{n\rightarrow\infty}a_n\leq1
    \end{equation*}
    We conclude that $ \lim_{n\rightarrow\infty}a_n=1$
    \myqed
\end{frame}


\begin{frame}
    \frametitle{Exercise-II$^*$}
    9. Prove that $\lim \sqrt[n]{n}=1$.
\end{frame}

\begin{frame}
    \frametitle{Solution-II$^*$}
    9.\\
    \hspace{1em} We consider the case for $n\geq 2$. Since $\sqrt[n]{n}>1$ for all $n\geq 2$, we can choose a
    non-negative sequence, let say $(b_n)_{n\geq 2}$ such that
    \begin{equation*}
        1+b_n=\sqrt[n]{n}
    \end{equation*}
    \hspace{1em} From the fact that
    \begin{equation*}
        (1+b_n)^n=\sum^n_{k=0}\left(\begin{smallmatrix}n\\k\end{smallmatrix}\right)1^k\cdot b_n^{n-k}
    \end{equation*}
    and another fact that $(1+b_n)^n=n$, we get the following inequality
    \begin{equation*}
        n=(1+b_n)^n\leq \left(\begin{smallmatrix}n \\ 2\end{smallmatrix}\right) b^2_n=\frac{n(n-1)}{2}b^2_n
    \end{equation*}
\end{frame}

\begin{frame}
    \frametitle{Solution-II$^*$}
    9.$(continue)$\\
    \hspace{1em} After some algebraic works, one can see
    \begin{equation*}
        b_n\leq\sqrt{\frac{2}{n-1}}
    \end{equation*}
    \hspace{1em} But this means
    \begin{equation*}
        0\leq \sqrt[n]{n}-1\leq \sqrt{\frac{2}{n-1}}
    \end{equation*}
    \hspace{1em} By taking the limit, we have
    \begin{equation*}
        0\leq \lim_{n\rightarrow\infty}(\sqrt[n]{n}-1)\leq \lim_{n\rightarrow\infty}\sqrt{\frac{2}{n-1}}=0
    \end{equation*}
\end{frame}

\begin{frame}
    \frametitle{Solution-II$^*$}
    9.$(continue)$\\
    \hspace{1em} Finally, by Squeeze theorem, we get
    \begin{equation*}
        \lim_{n\rightarrow\infty}(\sqrt[n]{n}-1)=\lim_{n\rightarrow\infty}\sqrt[n]{n}-1=0
    \end{equation*}
    \hspace{1em} We conclude
    \begin{equation*}
        \lim_{n\rightarrow\infty}\sqrt[n]{n}=1
    \end{equation*}
    \myqed
\end{frame}


\begin{frame}
    \frametitle{End}
    \vspace{2cm}
    \Huge \center  Have Fun and Learn Well!
\end{frame}

\end{document}
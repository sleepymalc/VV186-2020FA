\documentclass[12pt, t]{beamer}
\usepackage{graphicx}
\usepackage{amsmath}
\usepackage{setspace}
\usepackage{float} 
\usepackage{multido}
\usepackage{multirow}
\usepackage{array}
\usepackage{enumerate}
\usepackage{booktabs}
\usepackage{indentfirst} 
\usepackage[style=mla]{biblatex}
\usepackage{subcaption}
\usepackage{hyperref}
\usepackage{textpos}
\usepackage{mathtools, nccmath}

\makeatletter
\let\@@magyar@captionfix\relax
\makeatother

\definecolor{Turquoise3}{RGB}{0, 134, 139}
\renewcommand{\emph}[1]{{\color{Turquoise3}\textsl{#1}}}
\newcommand{\C}{\mathbb{C}} \newcommand{\F}{\mathbb{F}} \newcommand{\R}{\mathbb{R}} \newcommand{\Q}{\mathbb{Q}}
\newcommand{\N}{\mathbb{N}}
\newcommand{\myseries}[2]{$#1_1,#1_2,\dots,#1_#2$}
\newcommand{\nullspace}{~\\[15pt]}
\newcommand{\remark}{\textbf{Remark: }}
\newcommand{\scp}[2]{\langle\,#1\,,\,#2\,\rangle} \newcommand{\scpp}{\langle\,\cdot\,,\,\cdot\,\rangle}


\usetheme{Madrid}
\setbeamertemplate{navigation symbols}{}

\addtobeamertemplate{frametitle}{}{
\begin{textblock*}{100mm}(0.85\textwidth,-1cm)
\includegraphics[height=1cm]{Figures/logo/logo.png}
\end{textblock*}}

\definecolor{themecolor}{RGB}{25,25,112} 

\usecolortheme[named=themecolor]{structure}

\setbeamertemplate{items}[default]

\hypersetup{
    colorlinks=true,
    linkcolor=themecolor,
    filecolor=themecolor,      
    urlcolor=themecolor,
    citecolor=themecolor,
}

\title{VV186 RC Part II}
\subtitle{\textbf{From Numbers to Sequences}\\``We define everything as they now defined for a good reason..."}
\institute[UM-SJTU JI]{University of Michigan-Shanghai Jiao Tong University Joint Institute}
\author{Pingbang Hu}

\begin{document}

\begin{frame}
    \titlepage
    \begin{center}
        \includegraphics[height=2cm]{Figures/logo/logo2.png}
    \end{center}
\end{frame}

\begin{frame}
    \frametitle{Overview}
    \begin{enumerate}
        \item Natural Numbers \& Induction
        \item Rational Numbers
        \item Real Numbers
        \item Complex Numbers
        \item Sets and Points
        \item Interval
        \item Bound
        \item Exercises-I
        \item Functions and Maps
        \item Sequences
        \item Limits
        \item Exercises-II
    \end{enumerate}
\end{frame}

\section{Natural Numbers}
\begin{frame}
    \frametitle{Natural Numbers}
    The natural numbers can be \emph{constructed} from set theory, but that is not our main focus.
    Instead, we will simply denote the set of natural numbers by $\mathbb{N}$ and define it as:
    \begin{equation*}
        \mathbb{N}=\{0,1,2,3,4,\dots\}
    \end{equation*}
    We will have following properties we simply take it as \emph{axioms}.
    \begin{table}
        \centering
        \resizebox{11cm}{!}{%
            \begin{tabular}{ccc}
                \toprule
                $Porperties$     & \multicolumn{1}{c}{Addition}                         & \multicolumn{1}{c}{Multiplication}    \\
                \midrule
                $Associativity$  & $a+(b+c)=(a+b)+c$                                    & $a\cdot (b\cdot c)=(a\cdot b)\cdot c$ \\
                $Existence$      & $a+0=0+a=a$                                          & $a\cdot 1=1\cdot a=a$                 \\
                $Commutativity$  & $a+b=b+a$                                            & $a\cdot b=b\cdot a $                  \\
                $Distributivity$ & \multicolumn{2}{c}{$a\cdot (b+c)=a\cdot b+a\cdot c$}                                         \\
                \bottomrule
            \end{tabular}%
        }
    \end{table}
\end{frame}

\subsection{Induction}
\begin{frame}
    \frametitle{Mathematical Induction}
    There are two types of Mathematical Induction.\\
    \textbf{Mathematical Induction I \& II}\\
    \hspace{1em} Let $A$ be a statement that has to do with \textbf{natural numbers}. We denote
    the statement with respect to a specific number $n$ as $A(n)$. \\
    \vspace{1em}
    Then, \emph{type-I} induction works by establishing two statements:
    \begin{enumerate}
        \item $A(n_0)$ is true
        \item $A(n+1)$ is true whenever $A(n)$ is true for $n_0\leq n$
    \end{enumerate}
    \vspace{1em}
    And \emph{type-II} induction works by establishing similar two statements:
    \begin{enumerate}
        \item $A(n_0)$ is true
        \item $A(k+1)$ is true whenever $A(n)$ is true for all $n_0\leq n\leq k$
    \end{enumerate}
\end{frame}

\subsection{Induction}
\begin{frame}
    \frametitle{Induction}
    How, and when to use Induction?
    \begin{itemize}
        \item Examine the complexity of the problem, because using induction is sometimes much more
              complicated than using a direct method to prove a statement.
        \item Determine the initial condition (i.e. $n_0$) for your induction proof.
        \item Decide the part of the proof that you use induction and which induction you want to use.
              (There are few more different types of induction.)
        \item Make a short test of your method on draft paper to see whether it works and is easy to write down.
    \end{itemize}
\end{frame}

\begin{frame}
    \frametitle{Rational Numbers}
    We define that the set of rational numbers is
    \begin{equation*}
        \mathbb{Q}=\{\frac{p}{q}:p,q\in \mathbb{Z}\wedge q \neq 0\}
    \end{equation*}
    together with the following properties.
    \begin{table}
        \centering
        \resizebox{11cm}{!}{%
            \begin{tabular}{cccc}
                \toprule
                $Porperties$      & \multicolumn{1}{c}{Addition}                         &  & \multicolumn{1}{c}{Multiplication}    \\
                \midrule
                $Associativity$   & $a+(b+c)=(a+b)+c$                                    &  & $a\cdot (b\cdot c)=(a\cdot b)\cdot c$ \\
                \\
                $Neutral Element$ & $a+0=0+a=a$                                          &  & $a\cdot 1=1\cdot a=a$                 \\
                \\
                $Commutativity$   & $a+b=b+a$                                            &  & $a\cdot b=b\cdot a $                  \\
                \\
                $Inverse Element$ & $(-a)+a=a+(-a)=0$                                    &  & $a\cdot a^{-1}=a^{-1}\cdot a=1$       \\
                \\
                $Distributivity$  & \multicolumn{3}{c}{$a\cdot (b+c)=a\cdot b+a\cdot c$}                                            \\
                \bottomrule
            \end{tabular}%
        }
    \end{table}
\end{frame}

\begin{frame}
    \frametitle{Trichotomy law}
    There are still three \emph{axioms} we will define, and together with the nine axioms above,
    we build our rational numbers $\mathbb{Q}$.\\
    \vspace{1em}
    We assume that we know what a strictly positive rational number is, then we know we can find such
    a set $P$ with the property that :
    \begin{enumerate}
        \item $a=0$
        \item $a\in P$
        \item $-a\in P$
    \end{enumerate}
    which is so-called \textbf{trichotomy law}.\\
    Further more, we assume that the set of positive number $P$ is closed under addition and multiplication.\\
\end{frame}


\begin{frame}
    \frametitle{Real Numbers}
    We will adapt all axioms we had from rational numbers, except the last three axioms
    regard to trichotomy law. By replacing $P$ for $\mathbb{R}$, we are \textbf{almost}
    define our set of real numbers $\mathbb{R}$\\
    \vspace{1em}
    The last thing we need for $\mathbb{R}$ is listed below:\\
    \vspace{1em}
    If $A\subset \mathbb{R}$, $A\neq \emptyset$ is bounded above, then there exists a least upper
    bound for $A$ in $\mathbb{R}$.\\
    \vspace{1em}
    (Why we want to define real numbers?)
\end{frame}

\begin{frame}
    \frametitle{Complex Numbers}
    In Vv186, you just need to know how to perform basic complex numbers' computation and some basic properties.
    Here, we just list some basic computation rules and formulas.\\
    Given $z_1=(a_1,b_1)$ and $z_2=(a_2,b_2)$,
    \begin{itemize}
        \item $z_1+z_2=(a_1,b_1)+(a_2,b_2)=(a_1+a_2,b_1,b_2)$
        \item $z_1\cdot z_2=(a_1,b_1)\cdot (a_2,b_2)=(a_1a_2-b_1b_2,a_1b_2-a_2b_1)$
        \item $c\cdot z_1=c(a_1,b_2)=(ca_1,cb_1), c\in \mathbb{R} $
        \item $\bar{z_1}=(a_1,-b_1)$
        \item $|z_1|=\sqrt{a_1^2+b_1^2}=\sqrt{z_1\bar{z_1}}$
        \item $|z_1|^2=z_1\bar{z_1}$
    \end{itemize}
\end{frame}

\begin{frame}
    \frametitle{Sets and Points}
    We can classify points with respect to a set in following way:
    \begin{itemize}
        \item Interior point
        \item Exterior point
        \item Boundary point
        \item Accumulation Point
    \end{itemize}
    (What's the main role for all these definitions?)
\end{frame}

\begin{frame}
    \frametitle{Interval}
    \textbf{Theorem 2.1} Let $E \subseteq \mathbb{R}$ be a proper subset of $\mathbb{R}$ with at least two
    points. If $\forall a \notin E$, either $(a,+\infty)\cap E$ or $(-\infty, a)\cap E$ is an
    empty set, then $E$ is an interval.\\
    \textbf{Proof:}\\
    \hspace{1em} First we note that if $E$ is an interval, then for any $x,y\in E$ with $x<y$,
    the set $[x,y]$ is in $E$.\\
    \hspace{1em} Then we prove Theorem 2.1 by contraposition. Suppose $E$ is not an interval,
    then there is some $x,y\in E$ with $x<y$, such that $[x,y]$ is not in $E$. This means there
    exists some point $a\in [x,y]$ such that $a\notin E$. Both the set $(a,+\infty)\cap E$ and
    $(\infty,a)\cap E$ are non-empty, since the first one contain $y$, and the second one contains $x$.
\end{frame}

\begin{frame}
    \frametitle{Bound}
    \begin{itemize}
        \item A bound is defined in the total set of a set $A$, i.e., if the total set of $A$
              is $\mathbb{Q}$, then the bound is in $\mathbb{Q}$; if it's $\mathbb{R}$, then the
              bound is in $\mathbb{R}$.
        \item Usually, in Vv186, we \emph{assume} the total set is $\mathbb{R}$
        \item A bound may not be an element in the $A$
        \item Therefore, if $A$ doesn't have a maximum (or minimum), then the least upper bound
              (or greatest lower bound) of $A$ is not in $A$.
    \end{itemize}
\end{frame}

\begin{frame}
    \frametitle{Bound}
    \textbf{Example:}\\
    \begin{enumerate}
        \item The set $A=(-\infty, a)$ is bounded above in $\mathbb{R}$ with $supA=a$. It isn't in $A$.
        \item The set $B=[b,+\infty)$ is bounded below in $\mathbb{R}$ with $infB=b$. It's in $B$ since $b$ is the minimum of $B$.
        \item The set $C=[c,d)\cup(e,f)$ is bounded above and below in $\mathbb{R}$, so it's bounded with $supC=f$, $infC=c$.
        \item The set $D=\{x\in \mathbb{Q}^+: x=\frac{1}{n}, n\in \mathbb{N}^*\}$ is bounded above in $\mathbb{Q}^+$, but not bounded
              below in $\mathbb{Q}^+$.
    \end{enumerate}
\end{frame}

\begin{frame}
    \frametitle{Exercises-I}
    1. When learning the axioms of rational number, one student found that the operation of subsets of a non-empty
    set $X$ is somewhat similar to that of rational number:\\
    \vspace{1em}
    If we regard $\cup$ as $+$, $\cap$ as $\cdot$, then the equation
    \begin{equation*}
        A\cap(B\cup C)=(A\cap B)\cup(A\cap C)
    \end{equation*}
    is just the distributivity law. Help him  check whether P1 -- P9 also hold for such operations.
\end{frame}

\begin{frame}
    \frametitle{Exercises-I}
    2.
    \begin{enumerate}
        \item Prove that for $a_i\in \mathbb{Q}, i\in \mathbb{N^*}$, and $n$ is a natural number,
              \begin{equation*}
                  |\sum^n_{i=1}a_i|\leq\sum^n_{i=1}|a_i|
              \end{equation*}
        \item Prove that $|a-c|\leq |a-b| +|c-b|$
    \end{enumerate}
\end{frame}

\begin{frame}
    \frametitle{Exercises-I}
    3. We define $\mathbb{R}$ as a set containing rational numbers and the limits of rational numbers.
    Let $E\subseteq \mathbb{R}$ a non-empty subset of that is bounded above in $\mathbb{R}$. Now we don't
    assume P13 is an axiom. We want to prove P13 with P1 -- P12 and other facts that we know:
    \begin{itemize}
        \item Let's call the set of form $(a,+\infty)$ or $[a,+\infty)$ \emph{up interval}. Prove that the
              set of $E$'s upper bound, denoted by $U$, is an \emph{up interval}.
        \item (\textbf{Theorem 2.2}) $E$ has a least upper bound.
    \end{itemize}
\end{frame}

\begin{frame}
    \frametitle{Exercise-I$^*$}
    4. Let $A$ be bounded set in $\mathbb{R}$ (which means that the total set is $\mathbb{R}$), for any $\epsilon>0$,
    there is an element $x$ in $A$ such that $|x-supA|<\epsilon$.
\end{frame}

\begin{frame}
    \frametitle{Function}
    \begin{center}
        What is a function?\\
    \end{center}
    There are some crucial properties should be mentioned when you want to describe a function, they are:
    \begin{itemize}
        \item Domain:\\
              $dom\ f:=\{x\in X: (x,y)\ satisfies\ the\ requirement\ of\ f\}$
        \item Co-domain(target set): \\
              $Y=\{all\ "y"s\ such\ that\ (x,y)\ satisfies\ the\ requirement\ of\ f\}$
        \item Range:\\
              $ran\ f=\{y\ is\ in\ the\ target\ set:\ y=f(x)\}$
        \item How does it \emph{map}?
    \end{itemize}
    (In general, which set is bigger? Co-domain or Range?)
\end{frame}

\begin{frame}
    \frametitle{Function}
    There are few ways to express a function:
    \vspace{2em}
    \begin{table}
        \centering
        \resizebox{11cm}{!}{%
            \begin{tabular}{cc}
                \toprule
                $Ways\ to\ express\ function$         & $Comment$                                   \\
                \midrule
                $f:X\rightarrow Y,x\longmapsto f(x) $ & Pay attention to the forms of arrows.       \\
                \\
                $f:X\rightarrow Y,f(x)=\dots $        & An intuitive way to express a function.     \\
                \\
                $f=\{(x,y):P(x,y)\}$                  & When hard to show the explicit form of $f$. \\
                \bottomrule
            \end{tabular}%
        }
    \end{table}
\end{frame}

\begin{frame}
    \frametitle{Function}
    Common Misunderstanding:\\
    \begin{center}
        Function is a \emph{graph}
    \end{center}
    \hspace{1em} Rather, one can view a function as a machine that sends each element in its domain to its target set.
    This is the passive view of function. The function is like an ''bow "that shoots an element in its domain to its
    target set. This is the active point view of a function.\\
    \vspace{1em}
    \hspace{1em} However, often it's intuitive (and useful) to express a function using graphs, especially when $f$ is an
    $\mathbb{R}$ to $\mathbb{R}$ function.
\end{frame}

\begin{frame}
    \frametitle{Function}
    Common Misunderstanding:\\
    \begin{center}
        The \emph{co-domain is range} of $f$
    \end{center}
    \hspace{1em} The target set just contains $ranf$, it may be larger than $ranf$. For example, we can define a function
    \begin{equation*}
        g: \mathbb{R}\longmapsto \mathbb{R}, g(x)=x^2
    \end{equation*}
    \hspace{1em} The range of $f$ is $[0,+\infty)$, but the target set can simply be $\mathbb{R}$.\\
    \vspace{4em}
    Comment. On the contrary, the domain of $f$ contains exactly all the elements that have assignment with an element in
    $f$'s target set.
\end{frame}

\begin{frame}
    \frametitle{Sequence}
    A sequence is defined as:
    \begin{equation*}
        a_{(\cdot)}:\mathbb{N}\rightarrow \mathbb{N}, \ n\longmapsto a_n
    \end{equation*}
    And we start from looking at some common misunderstanding of sequences.
\end{frame}

\begin{frame}
    \frametitle{Sequence}
    Common misunderstanding:
    \begin{center}
        A sequence may \emph{not} contain infinitely many terms.
    \end{center}
    \hspace{1em} When we say ''sequence" , we usually assume that it is infinite. If it's finite, i.e., it
    contains only $(a_0,a_1,a_2,a_3,\dots)$, we usually say it is a ''\emph{tuple} ". This means
    the ''$\Omega$ "on slides 114 is an \emph{infinite set}. Similarly, a subsequence of a sequence is also infinite.
\end{frame}

\begin{frame}
    \frametitle{Sequence}
    Common misunderstanding:
    \begin{center}
        A sequence is either convergent or divergent to \emph{infinity}.
    \end{center}
    \hspace{1em} Of course, if a sequence is not convergent, we say it's ''divergent ". However,
    it doesn't mean it diverge to infinity.\\
    \hspace{1em}A classical example is
    \begin{equation*}
        a_n:=(-1)^n
    \end{equation*}
\end{frame}

\begin{frame}
    \frametitle{Sequence}
    Important Results \& Theorems \& Comments.
    \begin{itemize}
        \item A convergent sequence is bounded. (Slides 122)
        \item A convergent sequence has precisely one limit. (Slides 124)
        \item (\textbf{Squeeze Theorem}) \\Let $(a_n),(b_n)$ and $(c_n)$ be real sequences with $a_n<c_n<b_n$
              for sufficiently large $n\in\mathbb{N}$. Suppose that $\lim\ a_n=\lim\ b_n=:a$. Then
              $(c_n)$ converges and $\lim\ c_n=a$. (Slide 127)\\
              \emph{Comment.} It is extremely useful for examining the convergence of a sequence that is bounded.
        \item Let $(a_n)$ be a convergent sequence with limit $a$. Then any subsequence of $(a_n)$
              is convergent with the same limit. (Slide 139)
        \item Every real sequence has a monotonic subsequence. (Slide 140)
    \end{itemize}
\end{frame}

\begin{frame}
    \frametitle{Sequence}
    \begin{itemize}
        \item If a sequence has an accumulation point $x$, then there is a subsequence that converge to this point $x$.
        \item (\textbf{Bolzano--Weierstraß})\\ Every bounded real sequence has an accumulation point. \\(Slide 144)\\
              \emph{Comment.} There are at least two proofs, which we will discuss later.
        \item Every monotonic and bounded (real) sequence is convergent. (Slide 135)\\
              \emph{Comment.} This result holds for sequence in any space with an ordering (otherwise it's
              strange to even define ''monotonic").
    \end{itemize}
\end{frame}

\begin{frame}
    \frametitle{\textbf{Bolzano--Weierstraß}}
    \textbf{Bolzano--Weierstraß} Every bounded real sequence has an accumulation point.
    \begin{enumerate}
        \item Proof--1: On Horst's Slides.
        \item Proof--2:
              Since$(a_n)$ is bounded, assume $-M\leq a_n\leq M$ for all n. Divide the interval $[-M,M]$ into 2 sections:$[-M,0],[0,M]$.\\
              One of the interval, denoted by $I^{(1)}$, must contain infinitely many ''$a_n$"s(otherwise $(a_n)$is finite). Choose an $a_{(n,1)}$
              in $I^{(1)}$. We bisect $I^{(1)}$ into two intervals, one of which, denoted by $I^{(2)}$ must contain
              infinitely many ''$a_n$ "s. Choose an $a_{(n,2)}$ in $I^{(2)}$ that is different from $a_{(n,1)}$ . By
              repeatedly doing this procedure, we find a subsequence $(a_{n,k})_{k\in\mathbb{N}}$ that converges.
    \end{enumerate}
\end{frame}

\begin{frame}
    \frametitle{Limit}
    Now we take a step back, see some simple results for limit.\\
    \begin{center}
        Suppose $(a_n)\rightarrow a$ in $\mathbb{R}$ and $(b_n)\rightarrow b$ in $\mathbb{R}$
    \end{center}
    \begin{enumerate}
        \item $\lim (a_n+b_n)=a+b$
        \item $\lim (a_n\cdot b_n) =a\cdot b$
        \item $\lim \frac{a_n}{b_n}=\frac{a}{b}, b\neq 0$
    \end{enumerate}
    We will prove 3 now.
\end{frame}

\begin{frame}
    \frametitle{Limit}
    \begin{center}
        $\lim \frac{a_n}{b_n}=\frac{a}{b}, b\neq 0$
    \end{center}
    \textbf{Proof:}\\
    \hspace{1em} We want to prove that $|\frac{a_n}{b_n}-\frac{a}{b}|\rightarrow 0$ as $n\rightarrow \infty$.
    Since we don't want $b_n$ to be zero, we fix some $M\in \mathbb{N}$ such that $|b_n-b|<\frac{1}{2}|b|$ for
    all $n>M$. Now, when $n>M$, we have:
    \begin{multline*}
        |\frac{a_n}{b_n}-\frac{a}{b}|
        = |\frac{a_nb-b_na}{b_nb}|=|\frac{a_nb-ab+ab-b_na}{b_nb}| \\
        \leq \frac{|a_n-a||b|}{|b_nb|}+\frac{|a||b-b_n|}{|b_nb|}
        <\frac{2|a_n-a||b|}{b^2}+\frac{2|a||b-b_n|}{b^2}
    \end{multline*}
\end{frame}

\begin{frame}
    \frametitle{Limit}
    \hspace{1em} Given $\epsilon >0$, choose $N>M$ such that
    \begin{equation*}
        \forall n> N, |a_n-a|<\frac{|b|\epsilon}{4}|b_n-b|<\frac{b^2}{|a|}\cdot \frac{\epsilon}{4}
    \end{equation*}
    then we have:
    \begin{equation*}
        \forall n>N, |\frac{a_n}{b_n}-\frac{a}{b}|<\frac{|b|\epsilon}{4}\cdot\frac{2|b|}{b^2}+\frac{2|a|}{b^2}\cdot \frac{b^2}{|a|}\cdot \frac{\epsilon}{4}<\epsilon
    \end{equation*}
\end{frame}

\begin{frame}
    \frametitle{Limit}
    Here we list some common results for limit:
    \begin{itemize}
        \item $\lim_{n\rightarrow \infty} (\frac{1}{n^\alpha})=0,\alpha \in (0,+\infty)$
        \item $\lim_{n\rightarrow \infty} \sqrt[n]{a}=1, a>0$
        \item $\lim_{n\rightarrow \infty} \sqrt[n]{n}=1$
    \end{itemize}
\end{frame}

\begin{frame}
    \frametitle{Exercise-II}
    5. A sequence is defined as
    \begin{equation*}
        (S_n)_{n\in\mathbb{N}}, S_1=\sqrt{2}, S_2=\sqrt{2\sqrt{2}}, S_3=\sqrt{2\sqrt{2\sqrt{2}}}
    \end{equation*}
    Please calculate the limit of $(S_n)$ as $n\rightarrow \infty$, if it exists.
\end{frame}

\begin{frame}
    \frametitle{Exercise-II}
    6. Let $(a_n),(b_n)$ be two real sequences. Furthermore, assume that $a_n<b_n$
    for all $n$, $[a_{n+1}, b_{n+1}]\subseteq [a_n,b_n]$, $\lim (a_n-b_n)=0$. Prove that there
    is an unique $m\in [a_n,b_n]$ for all $n$, such that
    \begin{equation*}
        \lim a_n=\lim b_n=m
    \end{equation*}
\end{frame}


\begin{frame}
    \frametitle{Exercise-II}
    7. Let $(x_n)$ be a bounded real sequence. Then define
    \begin{equation*}
        a_n:=\sup_{m\geq n}(x_m),\quad b_n:=\inf_{m\geq n}(x_m)
    \end{equation*}

    \begin{enumerate}
        \item Prove that $(a_n)$ is decreasing, while $(b_n)$ is increasing.
        \item Since both $(a_n),(b_n)$ are monotonic and bounded, they are convergent.
              We denote $\underline{\lim} x_n=\lim b_n$; $\overline{\lim}x_n=\lim a_n$. Show that:
              \begin{equation*}
                  \underline{\lim}y_n+\underline{\lim}z_n\leq \underline{\lim} (y_n+z_n)\leq\overline{\lim}y_n+\underline{\lim}z_n\leq\overline{\lim}y_n+\overline{\lim}z_n
              \end{equation*}
    \end{enumerate}
\end{frame}

\begin{frame}
    \frametitle{Exercise-II$^*$}
    8. Let $(a_n)$ be a sequence such that
    \begin{equation*}
        a_n=\frac{1}{\sqrt{n^2+1}}+\dots+\frac{1}{\sqrt{n^2+n}}
    \end{equation*}
    Calculate the limit of $(a_n)$.
\end{frame}

\begin{frame}
    \frametitle{Exercise-II$^*$}
    9. Prove that $\lim \sqrt[n]{n}=1$.
\end{frame}

\begin{frame}
    \frametitle{End}
    \vspace{2cm}
    \Huge \center  Have Fun and Learn Well!
\end{frame}

\end{document}
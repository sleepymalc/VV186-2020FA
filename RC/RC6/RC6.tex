\documentclass[12pt, t]{beamer}
\usepackage{graphicx}
\usepackage{amsmath}
\usepackage{setspace}
\usepackage{float} 
\usepackage{multido}
\usepackage{multirow}
\usepackage{array}
\usepackage{enumerate}
\usepackage{booktabs}
\usepackage{indentfirst} 
\usepackage[style=mla]{biblatex}
\usepackage{subcaption}
\usepackage{hyperref}
\usepackage{textpos}
\usepackage{mathtools, nccmath}

\makeatletter
\let\@@magyar@captionfix\relax
\makeatother

\definecolor{Turquoise3}{RGB}{0, 134, 139}
\renewcommand{\emph}[1]{{\color{Turquoise3}\textsl{#1}}}
\newcommand{\C}{\mathbb{C}} \newcommand{\F}{\mathbb{F}} \newcommand{\R}{\mathbb{R}} \newcommand{\Q}{\mathbb{Q}}
\newcommand{\N}{\mathbb{N}}
\newcommand{\myseries}[2]{$#1_1,#1_2,\dots,#1_#2$}
\newcommand{\nullspace}{~\\[15pt]}
\newcommand{\remark}{\textbf{Remark: }}
\newcommand{\scp}[2]{\langle\,#1\,,\,#2\,\rangle} \newcommand{\scpp}{\langle\,\cdot\,,\,\cdot\,\rangle}


\usetheme{Madrid}
\setbeamertemplate{navigation symbols}{}

\addtobeamertemplate{frametitle}{}{
\begin{textblock*}{100mm}(0.85\textwidth,-1cm)
\includegraphics[height=1cm]{Figures/logo/logo.png}
\end{textblock*}}

\definecolor{themecolor}{RGB}{25,25,112} 

\usecolortheme[named=themecolor]{structure}

\setbeamertemplate{items}[default]

\hypersetup{
    colorlinks=true,
    linkcolor=themecolor,
    filecolor=themecolor,      
    urlcolor=themecolor,
    citecolor=themecolor,
}

\title{VV186 RC Part IV}
\subtitle{\textbf{Integration}\\``When learning integration, integrate all knowledge you have\dots"}
\institute[UM-SJTU JI]{University of Michigan-Shanghai Jiao Tong University Joint Institute}
\author{Pingbang Hu}

\begin{document}

\begin{frame}
    \titlepage
    \begin{center}
        \includegraphics[height=2cm]{Figures/logo/logo2.png}
    \end{center}
\end{frame}

\begin{frame}
    \frametitle{Overview}
    \begin{enumerate}
        \item Motivation
        \item step Function
        \item Regulated Integral
        \item Darboux Integral
        \item Riemann Integral
        \item Exercise
    \end{enumerate}
\end{frame}

\section{Motivation}
\begin{frame}
    \frametitle{Motivation}
    The motivation for integration is simple:
    \begin{itemize}
        \item Find area under curves.
        \item Find signed area for real life application.
        \item Do different kinds of \emph{transformation}.
        \item Solving equations.
    \end{itemize}
\end{frame}

\section{Step Function}
\begin{frame}
    \frametitle{Step Function}
    \begin{itemize}
        \item Partition
        \item Properties of step functions
              \begin{enumerate}
                  \item If $\varphi$ is a step function on $[a,b]$, then $\forall\alpha\in\mathbb{R}$, $\alpha\varphi$ is also a step function on $[a,b]$.
                  \item If $\varphi,\Psi$ are two step functions on $[a,b]$, then $\varphi+\Psi$ is also a step function on $[a,b]$.
                  \item The set consisting of step functions on $[a,b]$ where $a,b\in\mathbb{R}$ and $a<b$ is a \textbf{vector space}.(Why?)
              \end{enumerate}
    \end{itemize}


\end{frame}

\section{Step Function}
\begin{frame}
    \frametitle{Step Function}
    Properties of step function integral
    \begin{itemize}
        \item Given a step function $\varphi:[a,b]\rightarrow\mathbb{R}$, its integral exists and doesn't depend on the choice of partition.
        \item Let $T:Step([a,b])\rightarrow\mathbb{R}$ be a function (functional) and $T(\varphi)=\int_{a}^{b}\varphi$, then $T$ is a linear function that maps non-negative step functions to non-negative values.
        \item Let $\varphi\in[a,b]$, then $|\int^{b}_{a}\varphi|\leq\int^{b}_{a}|\varphi|$
    \end{itemize}
    \mbox{}
    \vfill
    \footnotesize($T$ is sometimes called "positive linear functional'')
\end{frame}

\section{Regulated Integral}
\begin{frame}
    \frametitle{Regulated Integral}
    \hspace{2em} Let $f\in\text{Reg}([a,b])$ and $(\varphi_n)$ a sequence in $\text{Step}([a,b])$ converging uniformly to $f$. Then the regulated integral of $f$, defined by
    \begin{equation*}
        \int_a^bf:=\lim\int_a^b\varphi_n
    \end{equation*}
    exists and does not depend on the choice of $(\varphi_n)$.
\end{frame}

\section{Darboux Integral}
\begin{frame}
    \frametitle{Darboux Integral}
    \hspace{2em} Let $[a,b]\subseteq\mathbb{R}$ be a closed interval and $f$ a bounded real function on $[a,b]$. Let $u_f$ denote the set of all step functions $u$ on $[a,b]$
    such that $u\geq f$ and denote $l_f$ the set of all step functions $l$ on $[a,b]$ such that $f\geq l$. The function $f$ is then said to be Darboux-integrable if
    \begin{equation*}
        \underline{\mathbb{I}}(f)=\sup_{l\in l_f}\int_a^bl=\inf_{u\in u_f}\int_a^bu=\overline{\mathbb{I}}(f)
    \end{equation*}

    \mbox{}
    \vfill\footnotesize
    We denote the integral of $f$ by $\mathbb{I}(f)$ to distinguish with the lower step functions $I$.
\end{frame}

\section{Riemann Integral}
\begin{frame}
    \frametitle{Riemann Integral}
    \hspace{2em} Let$[a,b]\subseteq\mathbb{R}$ be a closed interval and $f$ a bounded real function on $[a,b]$. Then $f$ is Riemann-integrable with integral
    $\int_a^bf\in\mathbb{R}$ if for every $\epsilon>0$ there exists a $\delta>0$ such that for any tagged partition on $[a,b]$ with mesh size $m(P)<\delta$,
    \begin{equation*}
        \left\vert \sum^n_{k=1}f(\xi_k)(x_k-x_{k-1})-\int_a^b f \right\vert <\epsilon
    \end{equation*}

    \mbox{}
    \vfill\footnotesize
    \hspace{1em} Comment. It is also a usual way to define Riemann integral by Darboux integral. In fact, Riemann integral is usually \emph{just the Darboux integral}.
\end{frame}

\section{Results/Theorem \& Comment}
\begin{frame}
    \frametitle{Results/Theorem \& Comment}
    \hspace{2em}
    The following are some results / Theorems $\&$ comments for integrals. They apply to all the three kinds of integrals that we learn in Vv186.

    \begin{enumerate}
        \item The integral is a linear map that maps non-negative functions to non-negative values.\\
              \vspace{1em}
        \item Let $f,g$ be two regulated functions on $[a,b]$. Moreover, if $f\leq g$, then $\int_a^bf\leq\int_a^bg$.\\
              \footnotesize Comment. This can be proved by taking the limit of the step functions.
    \end{enumerate}

\end{frame}

\section{Results/Theorem \& Comment}
\begin{frame}
    \frametitle{Results/Theorem $\&$ Comment}
    \begin{enumerate}
        \item[3.] The integral of $f$ doesn't change if $f$ changes its value on a \textbf{finite} set and is still integrable.\\
            \vspace{1em}
        \item[4.] The integral of $f$ doesn't change if $f$ changes its value on a \textbf{countable} set and is still integrable. If we take the usual understanding of "length of interval'' to integrate.\\
            \footnotesize Comment. It is important that $f$ is \emph{integrable}.\\
            \footnotesize Comment. This set is usually called \emph{measured zero set}.\\
            \vspace{1em}
        \item[5.] The Riemann integral, Darboux integral and the regulated integral of $f$ exist and coincide if $f$ is regulated.
    \end{enumerate}
\end{frame}

\section{Exercise}
\begin{frame}
    \frametitle{Exercise}
    1. This exercise gives an example of using step function sequence to find the integral of a function. Calculate
    \begin{equation*}
        \int_0^1x^4.
    \end{equation*}
\end{frame}

\section{Exercise}
\begin{frame}
    \frametitle{Exercise}
    2.
    \begin{itemize}
        \item Check whether the function $J$ given by
              \begin{equation*}
                  J:[0,1]\rightarrow\mathbb{R}, \qquad J(x)=
                  \begin{cases}
                      1, & x=\frac{1}{n}, \\
                      0, & otherwise;
                  \end{cases}
              \end{equation*}
              is regulated.\\
              \vspace{1em}
        \item Check whether the function $J$ we defined in i) is Riemann integrable.
    \end{itemize}
\end{frame}

\section{Exercise}
\begin{frame}
    \frametitle{Exercise}
    3. Please judge whether the following statements are true or false.
    \begin{itemize}
        \item $f(x)=x^3+e^x$ on $[-1,1]$ is regulated.
        \item Since $f(x)=x^2$ is regulated on each $[-n,n]$, where $n\in\mathbb{N}$, $f$ is regulated on $\mathbb{R}$, because we can let $n\rightarrow\infty$.
        \item A continuous function is piecewise continuous.
        \item Let $f,g$ be two real-valued functions defined on $[0,1]$. Furthermore, assume $f-g=x$, then the equation \begin{equation*}
                  \int_0^1f-\int_0^1g=\frac{1}{2}
              \end{equation*} holds.
        \item Let $f\in\text{Reg}([0,1])$, let $g$ be a real-valued function. The $f\circ g$ is regulated.
    \end{itemize}
\end{frame}

\section{Exercise}
\begin{frame}
    \frametitle{Exercise}
    4. Now let's prove the assertion on Slide 543. Let $f$ be a bounded real function on $[a,b]$ and $f$ is Darboux integrable. Please show that it is Riemann integrable on $[a,b]$.

\end{frame}

\section{Exercise}
\begin{frame}
    \frametitle{Exercise}
    5. Let $f$ be a piecewise continuous function on $[a,b]$. Prove that $f$ is regulated, i.e., for any $\epsilon>0$, there exists a step function $\varphi$ such that
    \begin{equation*}
        \sup_{x\in[a,b]}|f(x)-\varphi(x)|<\epsilon
    \end{equation*}
\end{frame}

\section{Exercise}
\begin{frame}
    \frametitle{Exercise}
    6. This exercise aims at showing two more characteristics of $\text{Reg}([a,b])$. Let $f,g\in\text{Reg}([a,b])$,
    \begin{enumerate}
        \item Show that the $f\cdot g\in\text{Reg}([a,b])$
        \item Show that $f^2\in\text{Reg}([a,b])$
        \item Suppose $f\geq p>0$ for some $p\in \mathbb{R}$. Show that $\frac{1}{f}\in\text{Reg}([a,b])$.
    \end{enumerate}
\end{frame}

\section{Exercise}
\begin{frame}
    \frametitle{Exercise}
    7. Give an example such that $\int_a^b|f|$ is Riemann integrable but $\int_a^bf$ is not Riemann integrable. Is the converse true?
\end{frame}

\section{Exercise}
\begin{frame}
    \frametitle{Exercise}
    8. Let $f:[a,b]\rightarrow\mathbb{R}$, $f$ is monotonic on $[a,b]$. Prove that $f$ is regulated. Is $f$ integrable?
\end{frame}

\section{Exercise*}
\begin{frame}
    \frametitle{Exercise*}
    9. Given $\epsilon>0$. For any regulated function $f:[0,1]\rightarrow\mathbb{R}$. There exists a continuous function $g:[0,1]\rightarrow\mathbb{R}$
    such that
    \begin{equation*}
        \left\vert \int_0^1f-\int_0^1g \right\vert <\epsilon\qquad\text{and}\qquad |f(x)-g(x)|<\epsilon\text{ on }[0,1]
    \end{equation*}
    except for a finite number of intervals with total length less than $\epsilon$.
\end{frame}

\begin{frame}
    \frametitle{End}
    \vspace{2cm}
    \Huge \center  Any Question?
\end{frame}

\begin{frame}
    \frametitle{End}
    \vspace{2cm}
    \Huge \center  Have Fun and Learn Well!
\end{frame}

\end{document}